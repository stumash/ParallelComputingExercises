\documentclass[11pt, letterpaper]{article}

\title{Assignment 3}
\author{
    Stuart Mashaal\\
    \texttt{260639962}
    \and
    Oliver Tse Sakkwun\\
    \texttt{260604362}
}
\date{Due: December 4, 2018}

\usepackage[utf8]{inputenc} % utf-8 file encoding
\usepackage[margin=0.75in]{geometry} % widen page margins
\usepackage{minted} % code samples
\setminted{fontsize=\small}
\usepackage{graphicx} % use external images
\graphicspath{ {images/} } % folder of images
\setlength\parindent{0pt} % no paragraph indent
\usepackage{amsmath} % math tools
\usepackage{enumitem} % convenient list formatting

\newcommand{\code}[1] { \mintinline{python3}{#1} }

\begin{document}

\begin{titlepage}
    \maketitle
    \thispagestyle{empty}
    \setcounter{page}{0}
\end{titlepage}

\section*{Question 1}
\label{sec:question_1}

\subsection*{1.1}
\label{sub:1_1}

\subsection*{1.2}
\label{sub:1_2}

\subsection*{1.3}
\label{sub:1_3}

\subsection*{1.4}
\label{sub:1_4}

\subsection*{1.5}
\label{sub:1_5}

\section*{Question 2}
\subsection*{2.1}
See in code
\subsection*{2.2}
The contains method scans the list to find the pair of nodes (pred,curr) reachable from head such that 
pred.next == curr,pred.key < key and curr.key >= key.The traversal uses hand-over-hand locking.
\begin{itemize}
\item Item is not in the list \\
When curr.key == key is false that is curr.key > key. From the sortedness 
invariant of the list, pred.next == curr, and pred.key < key we conclude that item cannot be in the 
set.

 \item Item is in the list \\
When curr.key == key is true, then from the uniqueness of keys that curr.item = item. Hence, item is in the set.
\end{itemize}


\section*{Question 3}
\subsection*{3.1}

\subsection*{3.2}


\section*{Question 4}
\subsection*{4.1}

\subsection*{4.2}

\subsection*{4.3}

\subsection*{4.4}

\end{document}
