\documentclass[11pt, letterpaper]{article}

\title{ECSE 420 Assignment 2 Report\\Group 19}
\author{
    Stuart Mashaal\\
    \texttt{260639962}
    \and
    Oliver Tse Sakkwun\\
    \texttt{260604362}
}
\date{Due: November 7, 2018}

\usepackage[utf8]{inputenc} % utf-8 file encoding
\usepackage[margin=0.75in]{geometry} % widen page margins
\usepackage{minted} % code samples
\usepackage{graphicx} % use external images
\graphicspath{ {images/} } % folder of images
\setlength\parindent{0pt} % no paragraph indent
\usepackage{amsmath} % math tools
\usepackage{enumitem} % convenient list formatting

\newcommand{\code}[1] { \mintinline{python3}{#1} }

\begin{document}

\begin{titlepage}
    \maketitle
    \thispagestyle{empty}
    \setcounter{page}{0}
\end{titlepage}

\section*{Question 1}
\label{sec:question_1}

\subsection*{1.2}
\label{sub:1_2}

Some stuff

\subsection*{1.4}
\label{sub:1_4}

Some stuff

\subsection*{1.5}
\label{sub:1_5}

Some stuff

\newpage
\section*{Question 2}
\label{sec:question_2}

\subsection*{2.1 - LockOne}
\label{sub:2_1_lockone}

\textbf{No.} If the shared \code{flag} atomic registers are replaced by regular registers, \code{LockOne} does not still provide mutual exclusion. \textbf{Proof:}

\begin{figure*}[h!]
    \begin{minipage}{0.5\textwidth}
        \centering
        \textbf{Thread 1}
        \begin{minted}[linenos]{java}
    public void lock() {
        int i = ThreadID.get();
        int j = i - 1;
        flag[i] = true // only local
    //
    //
    //
    //
    //
    //
        while (flag[j]) {} // false
    } // enter critical section
        \end{minted}
    \end{minipage}
    \hspace{.5cm}
    \begin{minipage}{0.5\textwidth}
        \centering
        \textbf{Thread 2}
        \begin{minted}[linenos]{java}
    //
    //
    //
    //
    public void lock() {
        int i = ThreadID.get();
        int j = i - 1;
        flag[i] = true // only local
        while (flag[j]) {} // false
    } // enter critical section
    //
    //
        \end{minted}
    \end{minipage}
\end{figure*}

The problem is that if we don't use atomic registers, then the writes to \code{flag} are not necessarily shared right away. This means that both threads can write to \code{flag} without the other being able to detect it. Then, both can enter the critical section.

\subsection*{2.2 - LockTwo}
\label{sub:2_2_locktwo}

\textbf{No}, if shared variable \code{victim} uses a regular register instead of an atomic one \code{LockTwo} does not still provide mutual exclusion, for similar reasons to those of $2.1$. \textbf{Proof:}

\setminted{fontsize=\small}
\begin{figure*}[h!]
    \begin{minipage}{0.5\textwidth}
        \centering
        \textbf{Thread 1}
        \begin{minted}[linenos]{java}
    public void lock() {
      int i = ThreadID.get();
      victim = i //only local
    //
    //
    //
    //
    //
      while (victim == i){} //see t2 write
    } // enter critical section
        \end{minted}
    \end{minipage}
    \hspace{1cm}
    \begin{minipage}{0.5\textwidth}
        \centering
        \textbf{Thread 2}
        \begin{minted}[linenos]{java}
    //
    //
    //
    public void lock() {
      int i = ThreadID.get();
      victim = i //only local
      while (victim == i){} //see t1 write
    } //enter critical section
    //
    //
            \end{minted}
    \end{minipage}
\end{figure*}

Since the \code{victim} register is no longer atomic, writes to it by either thread are not only \textit{not shared immediately} but also are \textit{not sequentially consistent}. The Java Concurrency Model does not guarantee sequential consistency without use of synchronization primitives (such as the \code{volatile} keyword, which makes a register atomic).

\newpage
\section*{Question 3}
\label{sec:question_3}

Some stuff

\end{document}
